\chapter{Wnioski i możliwe dalsze usprawnienia}
\label{cha:wnioski}

Podczas tworzenia rozwiązania zaproponowanego jako część powyższej pracy, zostało napotkanych mnóstwo przeszkód oraz przypadków brzegowych w których algorytm nie funkcjonował poprawnie. Większość tych przypadków była spowodowana brakiem kontekstu geograficznego przy łączeniu punktów na mapie podczas poszukiwania przyległości wierzchołków w procesie tworzenia grafu używanego dalej przy wyszukiwaniu optymalnych tras. Przy założeniu wyszukiwania przyległości w odległości 30 metrów, algorytm nie był w stanie stwierdzić czy pomiędzy punktami nie znajduje się naturalna przeszkoda jak na przykład płot lub ciek wodny. Także poza prostym przewidywaniem nie jest w stanie stwierdzić, czy łączone wierzchołki znajdują się na tej samej wysokości lub nie ma pomiędzy nimi przepaści, tutaj za przykład można podać wykrywanie skrzyżowań pomiędzy trasą znajdującą się na moście oraz drogą przebiegającą pod nim. Dodatkowym problemem był także brak pewności co do poprawności wag przyznawanych każdej z dróg w procesie tworzenia grafu. Dane te były dobierane na podstawie eksperymentów i wielu iteracji testów przeprowadzonych podczas tworzenia rozwiązania. \newline
Jedynym sposobem na pewność że trasy są poprawnie ze sobą połączone oraz mają poprawnie przypisane wagi jest stworzenie narzędzia do ręcznego wprowadzania tras oraz łączenia ich wierzchołkami które będzie jednocześnie prezentowało je w czasie rzeczywistym zaznaczone na mapie. W ten sposób, dodając do procesu prosty sposób edycji wprowadzonych danych dla przypadków brzegowych oraz wykluczając maszynowe tworzenie wierzchołków, można uzyskać pewność że użytkownik zostanie w każdym przypadku poprawnie nawigowany na trasie. W analizowanych zbiorze danych geo-przestrzennych znajduje się 1040 dróg, przeniesienie ich przez człowieka przy użyciu stworzonego dedykowanego rozwiązania nie powinno być czasochłonne. \newline
Dodatkowym zyskiem z zastosowania opisanego wcześniej usprawnienia jest możliwość nadania użytkownikom prawa do samodzielnej edycji danych znajdujących się w grafie. Wymaganie to jest spowodowane bardzo rzadką aktualizacją danych przez miasto Kraków w portalu zikit.carto.com. W momencie pisania pracy, ostatnie aktualizacja zbioru dróg rowerowych miała miejsce w lipcu roku 2018 co niesie za sobą konsekwencje braku wielu tras istniejących już na mapie miast Kraków. Jednym z dalszych proponowanych usprawnień działania projektu jest także dodanie strony umożliwiającej wprowadzanie przez użytkowników nowych tras, połączonych z istniejącą infrastrukturą, które mogą być akceptowane i dołączane do zbioru tras w grafie przez zarządcę systemu z poziomu panelu administracyjnego. \newline
Z punktu widzenia analizy zastosowanych algorytmów trasowania, rozwiązaniami bliskimi optymalnemu są zarówno algorytmy A* jaki NBA, reszta analizowanych algorytmów nie gwarantowała optymalnego rozwiązania, co jest w przypadku analizowanego systemu parametrem kluczowym, lub działała zdecydowanie zbyt wolno aby mieć realne zastosowanie. Obydwa wymienione algorytmy gwarantują czas przeszukiwania grafu w celu znalezienia najkrótszej ścieżki w czasie który nie obciąża zbytnio serwera i jest komfortowy dla użytkownika w kontekście czasu ładowania się strony internetowej zawierającej trasę. Różnice w zastosowaniu obydwu wymienionych algorytmów mają znaczenie tylko w zastosowaniu globalnym, gdy z zaproponowanej aplikacji serwerowej będzie korzystać znaczna ilość użytkowników. W takim wypadku zmniejszenie kosztów utrzymania infrastruktury nawet o kilka procent gwarantuje znaczne zyski. W przypadku analizy algorytmów dla dużej skali działania oprogramowania, lepszy rozwiązaniem wydaje się być algorytm A* w swojej standardowej odmianie. W kontekście czasu obliczeń algorytm NBA nie gwarantuje znacznej poprawy działania, rezerwuje za to zdecydowanie większe zasoby procesora maszyny na której jest egzekwowany co sprawia że ustępuje na tym polu algorytmowi A*. Dla skali rozwiązania zaproponowanego w powyższym opracowaniu, algorytmy są sobie równe.
