\chapter{Streszczenie}
\label{cha:streszczenie}

W niniejszej pracy dyplomowej podjęto się stworzenia aplikacji nawigującej dla rowerzystów. Podstawowym założeniem było aby rozwiązanie to dostarczało – w miarę możliwości – optymalnych tras, a także by objęło swym działaniem jak największy obszar. W tym celu posłużono się danymi dostarczanymi przez krakowski magistrat, a dokładnie przez jednostkę odpowiedzialną za infrastrukturę, to jest ZTP. 
Zakres prac nad aplikacją obejmował:
\begin{itemize}
\item wygenerowanie, na podstawie dostępnych danych o ścieżkach rowerowych, spójnego grafu relacji tychże ścieżek
\item zaimplementowanie odpowiedniego algorytmu wyszukiwania najkrótszej ścieżki i zoptymalizowania go do jej potrzeb
\item stworzenie prostego i przyjaznego interfejsu użytkownika z wizualizacją trasy (wersja webowa) oraz nawigowaniem po niej (wersja mobilna)
\item porównanie i analiza działania aplikacji w przypadku implementacji różnych algorytmów wyszukiwania najkrótszej ścieżki
\end{itemize}
Powyższe działania wymagały znajomości oraz umiejętności praktycznego użycia takich zagadnień jak bazy danych, grafy, algorytmy wyszukiwania najkrótszej ścieżki, programowanie na platformy webowe oraz mobilne (iOS), a także analiza danych. Każda z tych dziedzin była w mniejszym lub większym stopniu potrzebna (lub co najmniej przydatna) przy realizacji kolejnych etapów niniejszego projektu.
Wynikiem końcowym prac jest gotowa aplikacja, która szybko wyznacza optymalną trasę rowerową po Krakowie pomiędzy zadanymi punktami oraz nawiguje po niej, a w przypadku zboczenia z obranej ścieżki wylicza nową. Udało się tym samym zrealizować założenia projektowe, a także przeanalizować wyniki w odniesieniu do różnych algorytmów.
Największą trudnością w toku prac nad niniejszą aplikacją, było stworzenie odpowiedniego grafu relacji ścieżek, co de facto wynikało ze specyfiki dostarczanych przez ZIKiT danych.